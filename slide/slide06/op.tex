\section{上机操作}
\begin{frame}[fragile]
\begin{free}[问题1]{}
  编写程序,创建一个具有26个元素的数组,并在其中存储26个小写字母,然后利用该数组打印26个大写字母。
\end{free}
\end{frame}


\begin{frame}[fragile,allowframebreaks]
\lstinputlisting[]{slide06/code_op/ex01.c}
\end{frame}


\begin{frame}[fragile]
\begin{free}[问题2]{}
使用嵌套循环产生下列图案
\begin{lstlisting}
$
$$
$$$
$$$$
$$$$$
\end{lstlisting}
\end{free}
\end{frame}

\begin{frame}[fragile,allowframebreaks]
\lstinputlisting[]{slide06/code_op/ex02.c}
\end{frame}

\begin{frame}[fragile]
  \begin{free}[问题3]{}
    使用嵌套循环产生下列图案
\begin{lstlisting}
F
FE
FED
FEDC
FEDCB
FEDCBA
\end{lstlisting}
\end{free}
\end{frame}

\begin{frame}[fragile,allowframebreaks]
\lstinputlisting[]{slide06/code_op/ex03.c}
\end{frame}

\begin{frame}[fragile]
\begin{free}[问题4]{}
  编写程序,由用户输入一个大写字母,使用嵌套循环产生如下图案。
\begin{lstlisting}
    A
   ABA
  ABCBA
 ABCDCBA
ABCDEDCBA
\end{lstlisting}
\end{free}
\end{frame}

\begin{frame}[fragile,allowframebreaks]
\lstinputlisting[]{slide06/code_op/ex04.c}
\end{frame}

\begin{frame}[fragile]
\begin{free}[问题5]{}
  编写一个程序打印一个数,表的每一行都给出一个整数、它的平方以及立方。要求用户输入表的上限和下限,使用一个for循环。
\end{free}
\end{frame}


\begin{frame}[fragile,allowframebreaks]
\lstinputlisting[]{slide06/code_op/ex05.c}
\end{frame}


\begin{frame}[fragile]
\begin{free}[问题6]{}
  编写程序,把一个单词读入一个字符数组,然后反向打印出这个词。
\end{free}
\end{frame}


\begin{frame}[fragile,allowframebreaks]
\lstinputlisting[]{slide06/code_op/ex06.c}
\end{frame}


\begin{frame}[fragile]
\begin{free}[问题7]{}
  编写程序,要求输入两个浮点数,然后打印出二者的差值以及二者的乘积所得的结果。在用户输入非数字的输入之前程序循环处理每对输入值。
\end{free}
\end{frame}


\begin{frame}[fragile,allowframebreaks]
\lstinputlisting[]{slide06/code_op/ex07.c}
\end{frame}


\begin{frame}[fragile]
\begin{free}[问题8]{}
  改写以上程序,让它使用一个函数来返回计算值。
\end{free}
\end{frame}

\begin{frame}[fragile]
\begin{free}[问题9]{}
  编写程序,让用户输入一个下限整数和上限整数,然后依次计算从下限到上限的每一个整数的平方和。当上限小于或等于上限时,程序结束。运行结果如下:
\begin{lstlisting}
Enter lower and upper integer limit: 5 9
The sum of the squares from 25 to 81 is 255
Enter lower and upper integer limit: 3 25
The sum of the squares from 9 to 625 is 5520
Enter lower and upper integer limit: 5 5
Done
\end{lstlisting}
\end{free}
\end{frame}

\begin{frame}[fragile,allowframebreaks]
\lstinputlisting[basicstyle=\ttfamily\small]{slide06/code_op/ex09.c}
\end{frame}


\begin{frame}[fragile]
\begin{free}[问题10]{}
  编写程序,把8个整数读入一个数组,然后以相反顺序打印。
\end{free}
\end{frame}

\begin{frame}[fragile,allowframebreaks]
\lstinputlisting[]{slide06/code_op/ex10.c}
\end{frame}

\begin{frame}[fragile]
\begin{free}[问题11]{}
  编写程序,计算级数
$$
\begin{aligned}
&1+\frac12 + \frac13 + \frac14 + \cdots + \frac1n + \cdots\\
&1-\frac12 + \frac13 - \frac14 + \cdots + (-1)^{n+1}\frac1n + \cdots
\end{aligned}
$$
由用户输入要计算的项数$n$,观察$n=20,100,500$时的结果,判断级数是否会收敛。
\end{free}
\end{frame}

\begin{frame}[fragile,allowframebreaks]
\lstinputlisting[basicstyle=\ttfamily\small]{slide06/code_op/ex11.c}
\end{frame}


\begin{frame}[fragile]
\begin{free}[问题12]{}
  编写程序,创建一个长度为8的int数组,并把元素分别设为2的前8次幂,然后打印出它们的值。使用for循环来设置值,使用do while循环来显示值。
\end{free}
\end{frame}

\begin{frame}[fragile,allowframebreaks]
\lstinputlisting[]{slide06/code_op/ex12.c}
\end{frame}


\begin{frame}[fragile]
\begin{free}[问题13]{}
  编写程序,创建两个长度为8的double数组,使用一个循环让用户键入第一个数组的8个元素的值,并把第二个数组第$i$个元素设置为第一个数组的前$i$个元素之和。最后使用一个循环来显示两个数组的内容,第一个数组在一行中显示,而第二个数组中的每个元素在第一个数组的对应元素之下显示。
\end{free}
\end{frame}

\begin{frame}[fragile,allowframebreaks]
\lstinputlisting[]{slide06/code_op/ex13.c}
\end{frame}

\begin{frame}[fragile]
\begin{free}[问题14]{}
  编写程序,读入一行输入,然后反向打印该行。
\end{free}
\end{frame}

\begin{frame}[fragile,allowframebreaks]
\lstinputlisting[basicstyle=\ttfamily\small]{slide06/code_op/ex14.c}
\end{frame}


\begin{frame}[fragile]
\begin{free}[问题15]{}
 小王以每年10\%的单利投资了¥10000,而小李以每年5\%的复利投资了¥10000。编制一个程序,计算需要多少年小王的投资额才会超过小李,并显示那时两个人的投资额。
\end{free}
\end{frame}

\begin{frame}[fragile,allowframebreaks]
\lstinputlisting[basicstyle=\ttfamily\small]{slide06/code_op/ex15.c}
\end{frame}


\begin{frame}[fragile]
\begin{free}[问题16]{}
  小王中奖100万美元,存入一个每年赢得8\%的账户。在每年的最后一天,小王取出10万美元。编写一个程序,计算需要多少年才会清空账户。
\end{free}
\end{frame}


\begin{frame}[fragile,allowframebreaks]
\lstinputlisting[]{slide06/code_op/ex16.c}
\end{frame}
