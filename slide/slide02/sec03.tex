\section{使程序可读的技巧}
\begin{frame}[fragile]\ft{提高程序可读性}
\begin{itemize}
\item 变量命名时做到“见其名知其意”;\\[0.1in]
\item 合理使用注释;\\[0.1in]
\item 使用空行分隔一个函数的各个部分,如声明、操作等;\\[0.1in]
\item 每条语句用一行。注意,C允许把多条语句放在同一行或一条语句放多行;\\[0.1in]
\item 建议在程序开始处用一个注释说明文件名和程序的作用。该过程花不了多少时间,但对以后浏览或打印程序很有帮助;\\[.1in]
\item 当程序比较复杂时,使用多个函数会可实现程序的模块化,使程序可读性更强。
\end{itemize}
\end{frame}


% \begin{frame}[fragile]\ft{提高程序可读性}
% \begin{lstlisting}[
% language=c,
% frame=single,
% numbers=left
% ]
% // mile_km.c: Convert 2 miles to kilometers
% #include<stdio.h>
% int main(void) 
% {
%   float mile, km;       
%   mile = 2;
%   km = 1.6 * mile;
%   printf("%d mile = %d km\n", mile, km);
%   printf("Yes, %d km\n", 1.6 * mile);
%   return 0;
% }
% \end{lstlisting}
% \end{frame}


% \begin{frame}[fragile]{\secname}

% \begin{lstlisting}[
% language=c,
% frame=single,
% % numbers=left
% ]
%   float mile, km;
% \end{lstlisting}
% 等同于
% \begin{lstlisting}[
% language=c,
% frame=single,
% % numbers=left
% ]
%   float mile;
%   float km;
% \end{lstlisting}
% \end{frame}

% \begin{frame}[fragile]\ft{\secname}

%   \begin{itemize}
%   \item 第一个 \lstinline|printf| 语句用了两个占位符:
%     \begin{itemize}
%     \item 第一个 \lstinline|%d| 为  \lstinline|mile| 占位
%     \item 第二个 \lstinline|%d| 为  \lstinline|km| 占位
%     \item 圆括号 \lstinline|()| 中有三个参数,之间用逗号隔开。
%     \end{itemize}

%   \item  第二个 \lstinline|printf| 语句说明输出的值可以是一个表达式。
%   \end{itemize}


% \end{frame}
