\section{while语句}

\begin{frame}[fragile]\ft{\secname}
\begin{lstlisting}[language=c,backgroundcolor=\color{red!10}]
while (condition)
  statement
\end{lstlisting}
\begin{lstlisting}[language=c,backgroundcolor=\color{red!10}]
while (condition)
{
  statements
}
\end{lstlisting}
\end{frame}

\begin{frame}[fragile]\ft{\secname}
\begin{figure}
\centering
\tikzstyle{startstop}=[rectangle,rounded corners,minimum width=3cm,minimum height=1cm,text centered,draw=black,fill=red!30]
\tikzstyle{process}=[rectangle,minimum width=3cm,minimum height=1cm,text centered,draw=black,fill=orange!30]
\tikzstyle{decision}=[diamond,aspect=2,minimum width=3cm,minimum height=.5cm,text centered,draw=black,fill=green!30]
\tikzstyle{arrow}=[thick,->,>=stealth]

\begin{tikzpicture}[node distance=2.5cm]
\node (start) [] {};
\node (dec) [decision,below of=start] {condition};
\node (pro) [process,below of=dec,align=left] 
{\{\\
~~~~~statements;\\
\}
};
\node (right) [right of=pro] {};

\draw [arrow] (start) -- (dec);
\draw [arrow] (dec) -- node[anchor=east] {yes} (pro);
\draw [thick] (pro) -- ++(4,0);
\draw [arrow] (pro)++(4,0)|- (dec);
\draw [arrow] (dec) -- node[anchor=south] {no}
node[align=left,anchor=north] {go to next  \\ statement} ++(-4,0);
\end{tikzpicture}

\end{figure}

\end{frame}

\begin{frame}[fragile]\ft{\secname:终止while循环}
构造一个while循环时,必须能改变判断表达式的值,并最终使其为假,否则循环永远不会终止。
\end{frame}

\begin{frame}[fragile]\ft{\secname:终止while循环}
\begin{lstlisting}[language=c]
index = 1;
while (index < 5)
{
  printf("Good morning!\n");
}
\end{lstlisting} 
\rule{\textwidth}{1mm}\pause \vspace{0.1in}

这一段代码无法终止循环,因为在循环中不能改变index的值。
\end{frame}

\begin{frame}[fragile]\ft{\secname:终止while循环}
\begin{lstlisting}[language=c]
index = 1;
while (--index < 5)
{
  printf("Good morning!\n");
}
\end{lstlisting} 
\rule{\textwidth}{1mm}\pause \vspace{0.1in}

虽然改变了index的值,但却朝着错误的方向,故仍无法退出循环。
\end{frame}

\begin{frame}[fragile]\ft{\secname:终止while循环}
\begin{lstlisting}[language=c]
index = 1;
while (++index < 5)
{
  printf("Good morning!\n");
}
\end{lstlisting} 
\rule{\textwidth}{1mm}\pause \vspace{0.1in}

这段代码可以正常退出循环。
\end{frame}

\begin{frame}[fragile]\ft{\secname:何时终止循环}
只有在计算判断条件的值时才能决定是否终止循环。
\end{frame}

\begin{frame}[fragile]\ft{\secname:何时终止循环}
  \begin{minipage}{0.65\textwidth}
\lstinputlisting[language=c,frame=single,numbers=left]{Code/when.c}    
  \end{minipage}~~~~
  \begin{minipage}{0.3\textwidth}
\begin{lstlisting}[backgroundcolor=\color{red!10}]
n = 5
Now n = 6
n = 6
Now n = 7
\end{lstlisting}
    
  \end{minipage}


\end{frame}

\begin{frame}[fragile]\ft{\secname:while:入口条件循环}
while循环是使用入口条件的有条件循环。
\end{frame}

\begin{frame}[fragile]\ft{\secname:何时终止循环}
\begin{lstlisting}[language=c]
index = 10;
while (index++ < 5)
  printf("Have a fair day or better.\n");
\end{lstlisting}
\rule{\textwidth}{1mm}\pause 

把第一行改为
\begin{lstlisting}
index = 3;
\end{lstlisting}
就可以执行这个循环了。
\end{frame}

\begin{frame}[fragile]\ft{\secname:语法要点}
在使用while时,请确定循环体的范围。缩进是为了帮助读者而不是计算机。
\end{frame}

\begin{frame}[fragile]\ft{\secname:语法要点}
  \begin{minipage}{0.6\textwidth}
\lstinputlisting[language=c,frame=single,numbers=left]{Code/while1.c}    
  \end{minipage} ~~~~\pause 
  \begin{minipage}{0.3\textwidth}
\begin{lstlisting}[backgroundcolor=\color{red!10}]
n = 0
n = 0
n = 0
n = 0
... 
\end{lstlisting}    
  \end{minipage}
\end{frame}

\begin{frame}[fragile]\ft{\secname:语法要点}
while语句在语法上算作一条单独的语句,即使它使用了复合语句。
该语句从while开始,到第一个分号结束;在使用了复合语句的情况下,到终结花括号结束。
\end{frame}

\begin{frame}[fragile]\ft{\secname:语法要点}
\lstinputlisting[language=c,frame=single,numbers=left]{Code/while2.c}
\pause 
\begin{lstlisting}[backgroundcolor=\color{red!10}]
n = 4
That's all this program does.
\end{lstlisting}
\end{frame}

\begin{frame}[fragile]\ft{\secname:语法要点}
在C语言中,\textcolor{acolor1}{单独的分号代表空语句(null statement)。}
\end{frame}

\begin{frame}[fragile]\ft{\secname:语法要点}
有些时候,程序员会有意地使用带空语句的while语句。例如,假定你想要跳过输入直到第一个不为空或数字的字符,可以这样做。
\rule{\textwidth}{1mm}\pause 

\begin{lstlisting}[language=c]
while(scanf("%d",&num)==1)
  ;
\end{lstlisting}\pause

请注意,为了清楚起见,请把分号单独置于while的下一行。
\end{frame}

