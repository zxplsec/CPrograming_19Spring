\section{输入确认}
\begin{frame}[fragile]\ft{\secname}
\begin{itemize}
\item
在实际情况中,用户并不总是遵循指令,在程序所希望的输入与其实际输入之间可能存在不匹配,这可能会导致程序运行失败。\\[0.1in]
\item
作为程序员,你应该预见所有可能的输入错误,修正程序以使其能检测到这些错误并作出处理。
\end{itemize}
\end{frame}

\begin{frame}[fragile]\ft{\secname}
1、如有一个处理非负数的循环,用户可能会输入一个负数,你可以用一个关系表达式来检测这类错误:
\begin{lstlisting}[language=c]
int n;
scanf("%d", &n); // get first value
while (n >= 0)   // detect out-of-range value
{
  // process n
  scanf("%d", &n); // get next value
}
\end{lstlisting}
\end{frame}

\begin{frame}[fragile]\ft{\secname}
2、当然用户还可能输入类型错误的值,如字符 \lstinline|q|。检测这类错误的方式是检测 \lstinline|scanf()| 的返回值。 \vspace{0.1in}

该函数返回成功读入的项目个数,因此仅当用户输入一个整数时,下列表达式为真:
\begin{lstlisting}[language=c]
scanf("%d", &n) == 1
\end{lstlisting}
\end{frame}

\begin{frame}[fragile]\ft{\secname}
考虑以上两种可能出现的输入错误,我们可以对代码进行改进:
\begin{lstlisting}[language=c]
int n;
while (scanf("%d", &n) == 1 && n >= 0) 
{
   // process n 
}
\end{lstlisting}
\lstinline|while|循环的条件是“当输入是一个整数并且该整数为正”。
\end{frame}

\begin{frame}[fragile]\ft{\secname}
上面的例子中,当输入类型有错时,则终止输入。而更合适的处理方式是让程序对用户更加友好,给用户尝试输入正确类型的机会。\vspace{.1in}

\begin{itemize}
\item 首先要剔除那些有问题的输入,因 \lstinline|scanf()| 没有成功读取输入,会将其留在输入队列中。\\[0.1in]
\item 然后使用 \lstinline|getchar()| 来逐个字符地读取输入。
\end{itemize}

\end{frame}

\begin{frame}[fragile]\ft{\secname}
  \begin{free}[例]{}
    编制程序,计算特定范围内所有整数的平方和。限制这个特定范围的上界不应大于$1000$,下界不应小于$-1000$。
  \end{free}

\end{frame}


\begin{frame}[fragile,allowframebreaks]\ft{\secname}
\lstinputlisting
[]
{slide08/code/checking.c}
\end{frame}

\begin{frame}[fragile]\ft{\secname}
  对于 \lstinline|get_int()|,
\begin{itemize}
\item 
该函数试图将一个 \lstinline|int| 值读入变量 \lstinline|input|。\\[0.1in]
\item 
若失败,则该函数进入外层 \lstinline|while| 循环,然后内层 \lstinline|while| 循环逐个字符地读取那些有问题的输入字符。\\[0.1in]
\item
然后该函数提示用户重新尝试。外层循环继续运行,直至用户成功地输入一个整数。
\end{itemize}

\end{frame}

\begin{frame}[fragile]\ft{\secname}
对于 \lstinline|bad_limits()|,用户输入一个下界和上界来定义值域。需要的检查可能有 \vspace{.1in}
\begin{itemize}
\item 第一个值是否小于等于第二个值;\\[0.1in]
\item 两个值是否在可接受的范围内。
\end{itemize}
\end{frame}

\begin{frame}[fragile,allowframebreaks]\ft{\secname}
  \begin{lstlisting}[basicstyle=\ttfamily\small]
This program computes the sum of the squares of integers in a range.
The lower bound should not be less than -1000 and
the upper bound should not be more than +1000.
Enter the limits (enter 0 for both limits to quit):
lower limit: 1q
upper limit: q is not an integer.
Please enter an integer value, such as 25, -178, or 3: 3
The sum of the squares of the integers from 1 to 3 is 14
Enter the limits (enter 0 for both limits to quit):
lower limit: q
q is not an integer.
Please enter an integer value, such as 25, -178, or 3: 3
upper limit: 5
The sum of the squares of the integers from 3 to 5 is 50
Enter the limits (enter 0 for both limits to quit):
lower limit: 4
upper limit: 3q
4 isn't smaller than 3.
Please try again.
Enter the limits (enter 0 for both limits to quit):
lower limit: q is not an integer.
Please enter an integer value, such as 25, -178, or 3: 0
upper limit: 0
Done.
\end{lstlisting}
\end{frame}



\begin{frame}[fragile]\ft{\secname:模块化编程}
使用独立的函数来实现不同的功能。\red{程序越大,模块化编程就越重要。} \vspace{0.1in}

\begin{itemize}
\item \lstinline|main()| 管理流程,为其它函数指派任务;\\[0.1in]
\item \lstinline|get_int()| 获取输入;\\[0.1in]
\item \lstinline|badlimits()| 检查值的有效性;\\[0.1in]
\item \lstinline|sum_squares()| 进行实际的计算。

\end{itemize}
\end{frame}

\begin{frame}[fragile]\ft{\secname:C输入的工作方式}
假如有输入
\begin{lstlisting}
is 28 12.4
\end{lstlisting}
在你看来,该输入是一串字符、一个整数、一个浮点值。而对C来说,该输入时一个字节流。\vspace{0.1in}

\begin{itemize}
\item 第1个字节是字母 \lstinline|i| 的字符编码\\[0.1in]
\item 第2个字节是字母 \lstinline|s| 的字符编码\\[0.1in]
\item 第3个字节是空格字符的字符编码\\[0.1in]
\item 第4个字节是数字 \lstinline|2| 的字符编码\\[0.1in]
\item ...
\end{itemize}
\end{frame}

\begin{frame}[fragile]\ft{\secname:C输入的工作方式}
当 \lstinline|getchar()| 遇到这一行,以下代码将读取并丢弃整行,包括数字,因为这些数字其实被看做是字符:
\begin{lstlisting}[language=c]
while((ch = getchar()) != '\n')
  putchar(ch);
\end{lstlisting}
\end{frame}

\begin{frame}[fragile]\ft{\secname:C输入的工作方式}
假如有输入
\begin{lstlisting}
42
\end{lstlisting}
在使用 \lstinline|scanf()| 函数时,不同的占位符会导致不同的效果。
\end{frame}

\begin{frame}[fragile]\ft{\secname:C输入的工作方式}
\begin{itemize}
\item 使用 \lstinline|%c|,将只读取字符 \lstinline|4| 并将其存储在一个char型变量中;\\[0.1in]
\item 使用 \lstinline|%s|,会读取两个字符,即字符 \lstinline|4| 和 \lstinline|2|,并将它们存储在一个字符串中\\[0.1in]
\item 使用 \lstinline|%d|,同样读取两个字符,但随后会计算与它们相应的整数值$4\times 10+2=42$,然后将该整数保存在一个 \lstinline|int| 变量中;\\[0.1in]
\item 使用 \lstinline|%f|,同样读取两个字符,计算对应的数值 \lstinline|42|,然后以浮点表示法表示该值,并将结果保存在一个 \lstinline|float| 型变量中。
\end{itemize}
\end{frame}
