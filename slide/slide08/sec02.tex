\section{缓冲区(Buffer)}

\begin{frame}[fragile]\ft{\secname} 
\begin{itemize}
\item    非缓冲输入\\[0.1in]
\item[]  立即回显:键入的字符对正在等待的程序立即变为可用
\begin{lstlisting}
HHeelllloo wwoorrlldd[enter]
II  aamm  hhaappppyy[enter]
\end{lstlisting}
\vspace{0.1in}

\item 缓冲输入\\[0.1in]
\item[] 延迟回显:键入的字符被存储在缓冲区中,按下回车键使字符块对程序变为可用。
\end{itemize}
\end{frame}

\begin{frame}[fragile]\ft{\secname:为什么需要缓冲区?}
\begin{itemize}
\item 将若干个字符作为一个块传输比逐个发送耗时要少。   \\[0.1in]
\item 若输入有误,可以使用键盘来修正错误。当最终按下回车键时,便可发送正确的输入。
\end{itemize}
\end{frame}
