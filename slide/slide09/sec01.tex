\section{函数概述}

\begin{frame}\ft{为什么使用函数?}
\begin{itemize}
\item 使用函数可以减少代码的重复。若程序需要多次使用某种特定的功能,只需编写一个合适的函数,然后程序可以在任何需要的地方调用该函数。\\[0.2in]
\item 即使某种功能在程序中只使用一次,将其以函数的形式实现也有必要,因为函数使得程序更加模块化,从而有利于程序的阅读、修改和完善。
\end{itemize}
\end{frame}

\begin{frame}\ft{为什么使用函数?}
假设你想编写一个程序,以实现如下功能:\vspace{0.1in}

\begin{itemize}
\item  读入一行数字 \\[0.1in]
\item  对数字进行排序 \\[0.1in]
\item  求他们的平均值 \\[0.1in]
\item  打印出一个柱状图
\end{itemize}
\end{frame}

\begin{frame}[fragile]\ft{为什么使用函数?}
\begin{lstlisting}[]
#include <stdio.h>
#define SIZE 50
int main(void)
{
  float list[SIZE];  
  readlist(list, SIZE);
  sort(list, SIZE);
  average(list, SIZE);
  bargragh(list, SIZE);  
  return 0;
}
\end{lstlisting}
\end{frame}

\begin{frame}[fragile]\ft{为什么使用函数?}
如何实现这四个函数需要你自行完成。描述性的函数名可以清楚地表明程序的功能和组织结构,然后对每个函数进行独立设计,若这些函数足够通用化,则可以在其他程序中调用它们。
\end{frame}

\begin{frame}[fragile]\ft{为什么使用函数?}
\begin{itemize}
\item
函数可看做是一个“黑盒子”,你只需关注函数的功能及使用方法,而其内部行为你无需考虑,除非你是该函数的编写者。\\[0.1in]
\item
如我们在使用 \lstinline|printf()| 时,只需输入一个控制字符串,或者还有其它一些参数,就可以预测 \lstinline|printf()| 的执行结果,而无须了解 \lstinline|printf()| 内部的代码。
\\[0.1in]
\item
以这种方式看待函数,有助于集中精力投入到程序的整体设计而不是实现细节。
\end{itemize}
\end{frame}

\begin{frame}[fragile]\ft{对于函数需要了解些什么?}
\begin{itemize}
\item 如何正确定义函数\\[0.1in]
\item 如何调用函数\\[0.1in]
\item 如何建立函数间的通信
\end{itemize}

\end{frame}

\begin{frame}[fragile]\ft{一个简单的例子}
请打印一个简单的信头:
\begin{lstlisting}[backgroundcolor=\color{red!10}]
****************************************
Wuhan University
299 Bayi Road Wuchang District,
Wuhan, PR China 430072
****************************************
\end{lstlisting}
\end{frame}

\begin{frame}[fragile,allowframebreaks]\ft{一个简单的例子}
\lstinputlisting
[]
{slide09/code/letterhead1.c}
\end{frame}


\begin{frame}[fragile]\ft{程序分析 }
  \lstinline|starbar()| 在不同位置出现了三次:\vspace{0.1in}

\begin{itemize}
\item 函数原型{\tf (function prototype)}:告知编译器starbar的函数类型\\[0.1in]
\item 函数调用{\tf (function call)}:使函数执行\\[0.1in]
\item 函数定义{\tf (function definition)}:实现函数的具体功能
\end{itemize}
\end{frame}

\begin{frame}[fragile]\ft{程序分析}

函数同变量一样有多种类型。 
函数在被使用之前都要声明其类型,故 \lstinline|main()| 之前出现了代码
\begin{lstlisting}[backgroundcolor=\color{red!10}]
void starbar(void);
\end{lstlisting}
\begin{itemize}
\item
圆括号表明 \lstinline|starbar()| 是一个函数名。\\[0.1in]
\item 
第一个 \lstinline|void| 指的是函数类型,表明该函数没有返回值。\\[0.1in]
\item 
第二个 \lstinline|void| 表明该函数不接受任何参数。\\[0.1in]
\item 
分号表示该语句是进行函数声明,而不是函数定义。 
\end{itemize}
\end{frame}

\begin{frame}[fragile]\ft{程序分析}
函数原型也可以放在main函数内变量声明的任何位置,故以下两种写法都正确:
\begin{lstlisting}[language=c,backgroundcolor=\color{red!10}]
...
void starbar(void);
int main(void)
{
  ...
}
\end{lstlisting}

\begin{lstlisting}[language=c,backgroundcolor=\color{red!10}]
...
int main(void)
{
  void starbar(void);
}
\end{lstlisting}

\end{frame}

\begin{frame}[fragile]\ft{程序分析}
程序在 \lstinline|main()| 中通过使用以下方式调用 \lstinline|starbar()| :
\begin{lstlisting}
starbar();
\end{lstlisting}
\begin{itemize}
\item
当程序执行到该语句时,它找到 \lstinline|starbar()| 并执行其中的指令。\\[0.1in]
\item 
执行完 \lstinline|starbar()| 中的代码后,程序将返回到调用函数{\tf (calling function)}的下一条语句继续执行。 
\end{itemize}
\end{frame}

\begin{frame}[fragile]\ft{程序分析}
\begin{itemize}
\item
程序中 \lstinline|starbar()| 和 \lstinline|main()| 有相同的定义格式,即首先以类型、名称和圆括号开始,接着是开始花括号、变量声明、函数语句定义以及结束花括号。\\[0.1in]
\item 
注意此处的 \lstinline|starbar()| 后跟花括号,告诉编译器这是在定义函数,而不是调用它或声明其原型。
\end{itemize}

\end{frame}

\begin{frame}[fragile]\ft{程序分析}
\begin{itemize}
\item
该程序中 \lstinline|starbar()| 和 \lstinline|main()| 在同一个文件中,也可以将它们放在不同文件中。\\[0.1in]
\item 
单文件形式比较容易编译,而使用多个文件则有利于在不同的程序中使用相同的函数。\\[0.1in]
\item 
若使用多文件形式,则每个文件中都必须包含 \lstinline|#define| 和 \lstinline|#include| 指令。
\end{itemize}

\end{frame}

\begin{frame}[fragile]\ft{程序分析}
\begin{itemize}
\item \lstinline|starbar()| 中的变量 \lstinline|count| 是一个局部变量,这意味着该变量只在 \lstinline|starbar()| 中可用。\\[0.1in]
\item 
即使你在其它函数中使用名称 \lstinline|count| ,也不会出现任何冲突。
\end{itemize}

\end{frame}

\begin{frame}[fragile]\ft{函数参数}
改写以上程序,让信头的文字居中,形如
\begin{lstlisting}
****************************************
            Wuhan University
    299 Bayi Road, Wuchang District,
         Wuhan, PR China 430072
****************************************
\end{lstlisting}
\end{frame}

\begin{frame}[fragile]\ft{如何做到?}
假设一行是40个字符宽度。\vspace{0.1in} \pause 

\begin{enumerate}
\item
打印一行星号很容易做到,直接输出40个星号即可。\\[0.1in] \pause 
\item 
如何让Wuhan University居中呢?。
\\[0.1in] \pause 
\item[]
在输出文字之前输出若干空格即可。\\[0.1in] \pause 
\item 
那到底输出多少个空格呢?。
\\[0.1in] \pause 
\item[]
设文字长度为$l$,则一行中除文字外还需$40-l$个空格。想要文字居中,左边应输出$(40-l)/2$个空格。
\end{enumerate}

\end{frame}

\begin{frame}[fragile,allowframebreaks]\ft{程序实现}
\lstinputlisting
[]
{slide09/code/letterhead2.c}
\end{frame}


\begin{frame}[fragile]\ft{定义带参数的函数(形式参数,简称“形参”)}
\begin{lstlisting}[language=c,backgroundcolor=\color{red!10},title=函数头]
  void show_n_char(char ch, int num)
\end{lstlisting}
\begin{itemize}
\item
这行代码告诉编译器,{\tf show\_n\_char()}使用了两个参数{\tf ch}和{\tf num},它们的类型分别为{\tf char}和{\tf int}。\\[0.1in]
\item
变量{\tf ch}和{\tf num}被称为形式参数{\tf (formal argument)}或形式参量{\tf (formal parameter)}。\\[0.1in]
\item 形式参量是局部变量,为函数所私有,这意味着可以在其它函数中使用相同的变量名。\\[0.1in]
\item 调用函数时,形式参量会被赋值。
\end{itemize}
\end{frame}

\begin{frame}[fragile]\ft{定义带参数的函数(形式参数,简称“形参”)}
必须在每个形参前声明其类型,不能像通常的变量声明那样使用变量列表来声明同一类型的变量。比如
\begin{lstlisting}[language=c,backgroundcolor=\color{red!10}]
void func1(int x, y, z)  // wrong 
void func2(int x, int y, int z)  // right
\end{lstlisting}

\end{frame}

\begin{frame}[fragile]\ft{定义带参数的函数(形式参数,简称“形参”)}
古老的函数定义方式1:
\begin{lstlisting}[language=c,backgroundcolor=\color{red!10}]
void show_n_char(ch, num)
char ch;
int num;
{
  ...
}
\end{lstlisting}
\end{frame}

\begin{frame}[fragile]\ft{定义带参数的函数(形式参数,简称“形参”)}
古老的函数定义方式2:
\begin{lstlisting}[language=c,backgroundcolor=\color{red!10}]
void func1(x, y, z)
int x, y, z;
{
  ... 
}
\end{lstlisting}
\end{frame}

\begin{frame}[fragile]\ft{带参数函数的声明}
\begin{itemize}
\item
使用函数之前需要用ANSI原型声明该函数
\begin{lstlisting}[language=c,backgroundcolor=\color{red!10}]
void show_n_char(char ch, int num);
\end{lstlisting}
\vspace{0.1in}

\item
当函数接受参数时,函数原型通过使用一个逗号分隔的类型列表指明参数的个数和类型。在函数原型中可根据你的喜好省略变量名:
\begin{lstlisting}[language=c,backgroundcolor=\color{red!10}]
void show_n_char(char, int);
\end{lstlisting}
\vspace{0.1in}

\item
在原型中使用变量名并没有实际地创建变量。
\end{itemize}

\end{frame}

\begin{frame}[fragile]\ft{带参数函数的声明}
ANSI C也支持旧的函数声明形式,即圆括号内不带任何参数:
\begin{lstlisting}[language=c,backgroundcolor=\color{red!10}]
void show_n_char();
\end{lstlisting}
该方式请不要使用。了解该形式的主要原因只是为了让你能正确识别并理解以前的代码。
\end{frame}

\begin{frame}[fragile]\ft{调用带参数的函数:实际参数,简称“实参”}
函数调用中,通过使用实际参数{\tf (actual argument)}对{\tf ch}和{\tf num}赋值。
\begin{itemize}
\item
第一次调用中
\begin{lstlisting}[language=c,backgroundcolor=\color{red!10}]
show_n_char(SPACE, 12);
\end{lstlisting}
实参是空格字符和{\tf 12},它们被赋给{\tf show\_n\_char()}中相应的形参:{\tf ch}和{\tf num}。\\[0.1in]
\item \red{实参可以是常量、变量或一个复杂的表达式。}\\[0.1in]
\item 但无论何种形式的实参,执行时首先要计算其值,然后将该值赋值给被调函数中相应的形参。
\end{itemize}
\end{frame}

\begin{frame}[fragile]\ft{调用带参数的函数:实际参数,简称“实参”}
实参赋值给形参,被调函数使用的值是从调用函数中复制而来的,故不管在被调函数中对赋值数值进行了什么操作,调用函数中的原数值不受影响。
\end{frame}

\begin{frame}[fragile]\ft{使用 \lstinline|return| 从函数中返回一个值}
\begin{itemize}
\item
将实参赋值给形参,实现了从调用函数到被调函数的通信。\\[0.1in]
\item
而想从被调函数往调用函数传递信息,可以使用函数返回值。
\end{itemize}

\end{frame}

\begin{frame}[fragile]\ft{使用 \lstinline|return| 从函数中返回一个值}
  \begin{free}[例]{}
    编写函数,比较两个整数的大小,并返回较小值。同时编制一个驱动程序来测试该函数。
  \end{free}
\end{frame}

\begin{frame}[fragile,allowframebreaks]\ft{使用 \lstinline|return| 从函数中返回一个值}
\lstinputlisting
[]
{slide09/code/lesser.c}
\end{frame}

\begin{frame}[fragile]\ft{使用 \lstinline|return| 从函数中返回一个值}

\begin{lstlisting}[backgroundcolor=\color{red!10}]
Enter two integers (q to quit):
509 333
The lesser of 509 and 333 is 333.
Enter two integers (q to quit):
-9333 6
The lesser of -9333 and 6 is -9333.
Enter two of integers (q to quit):
q
Bye.
\end{lstlisting}
\end{frame}

\begin{frame}[fragile]\ft{使用 \lstinline|return| 从函数中返回一个值}
\begin{itemize}
\item
关键字 \lstinline|return| 指明了其后的表达式的值即为该函数的返回值。\\[0.1in]
\item 
{\tf imin()}中的变量{\tf min}是其私有的,但 \lstinline|return| 语句将它的值返回给了调用函数。\\[0.1in]
\item 
语句
\begin{lstlisting}[language=c,backgroundcolor=\color{red!10}]
lesser = imin(m, n);
\end{lstlisting}
相当于把min的值赋给了lesser。
\\[0.1in]
\item
能否这么写?
\begin{lstlisting}[language=c,backgroundcolor=\color{red!10}]
imin(m, n);
lesser = min;
\end{lstlisting} \pause
{\Huge 当然不行!!!}
\end{itemize}
\end{frame}

\begin{frame}[fragile]\ft{使用return从函数中返回一个值}
返回值不仅可以被赋给一个变量,也可以被用作表达式的一部分。如
\begin{lstlisting}[language=c,backgroundcolor=\color{red!10}]
answer = 2*imin(m, n) + 5;
printf("%d\n", imin(answer+2, LIMIT));
\end{lstlisting}

\end{frame}

\begin{frame}[fragile]\ft{使用return从函数中返回一个值}
返回值可以由任何表达式计算而得到,而不仅仅来自于一个变量。如imin函数可以改写为
\begin{lstlisting}[language=c,backgroundcolor=\color{red!10}]
int imin(int n,int m)
{  
  return ((n < m) ? n : m);
}
\end{lstlisting}
这里并不要求使用圆括号,但如果想让程序更清晰,可以把添上一个圆括号。
\end{frame}

\begin{frame}[fragile]\ft{使用return从函数中返回一个值}
观察以下代码:
\begin{lstlisting}[language=c,backgroundcolor=\color{red!10}]
int what_if(int n)
{  
  double z = 100.0 / (double) n;
  return z;
}
\end{lstlisting}
这里,返回值的类型和声明的类型不一致,{\Large What will happen?}
\pause 
\vspace{0.1in}

\red{将把{\tf doule}型变量{\tf z}的值强制转换为{\tf int}型。}

\end{frame}

\begin{frame}[fragile]\ft{使用return从函数中返回一个值}
 \lstinline|return| 的另一个作用是终止函数的执行,并把控制返回给调用函数的下一条语句,即使 \lstinline|return| 语句不在函数尾部。如{\tf imin()}可以写成
\begin{lstlisting}[language=c,backgroundcolor=\color{red!10}]
int imin(int n, int m)
{  
  if (n < m)
    return n;
  else
    return m;  
  printf("Oh my god!\n");    
}
\end{lstlisting}
 \lstinline|return| 语句使得{\tf printf}语句永远不会执行。
\end{frame}

\begin{frame}[fragile]\ft{使用return从函数中返回一个值}
也可以使用语句
\begin{lstlisting}[language=c,backgroundcolor=\color{red!10}]
return;
\end{lstlisting}
该语句会终止执行函数,并把控制返回给调用函数。此时, \lstinline|return| 后没有任何表达式,故没有返回值,该形式只能用于 \lstinline|void| 类型的函数。
\end{frame}

\begin{frame}[fragile]\ft{函数类型}
\begin{itemize}
\item
函数应该进行类型声明,同时其类型应和返回值类型相同。\\[0.1in]
\item
无返回值的函数应该被声明为 \lstinline|void| 类型。\\[0.1in]
\item
类型声明是函数定义的一部分,该类型指的是返回值类型。如函数头
\begin{lstlisting}[language=c,backgroundcolor=\color{red!10}]
double klink(int a, int b)
\end{lstlisting}
表示函数使用两个{\tf int}型的参数,而返回值类型为{\tf double}。
\end{itemize}
\end{frame}

\begin{frame}[fragile]\ft{函数类型}
为正确使用函数,程序在首次调用函数之前需要知道该函数的类型。
\begin{itemize}
\item 方式一:
调用之前给出完整的函数定义。\\[0.1in]
\begin{lstlisting}[language=c,backgroundcolor=\color{red!10}]
int imin(int n, int m)
{
  ... 
}

int main(void)
{
  ...
  n = imin(n1, n2);
  ...
}
\end{lstlisting}
\end{itemize}
\end{frame}

\begin{frame}[fragile]\ft{函数类型}
\begin{itemize}
\item 方式二:
对函数进行声明,以便将函数信息通知编译器。\\[0.1in]
\begin{lstlisting}[language=c,backgroundcolor=\color{red!10}]
int imin(int, int);

int main(void)
{
  int n1, n2, lesser;
  ...
  n = imin(n1, n2);
  ...
}

int imin(int n, int m)
{
  ... 
}
\end{lstlisting}
\end{itemize}
\end{frame}

\begin{frame}[fragile]\ft{函数类型}
也可将函数声明放在调用函数内部。 
\begin{lstlisting}[language=c,backgroundcolor=\color{red!10}]
int main(void)
{
  int imin(int, int);
  int n1, n2, lesser;
  ...
  n = imin(n1, n2);
  ...
}

int imin(int n, int m)
{
  ... 
}
\end{lstlisting}
\end{frame}

\begin{frame}[fragile]\ft{函数类型}
在ANSI C标准库中,函数被分为几个系列,每一系列都有各自的头文件,这些头文件中包含了本系列函数的声明部分。
\end{frame}

\begin{frame}[fragile]\ft{函数类型}
  \begin{lstlisting}[language=c,backgroundcolor=\color{red!10}]
// stdio.h 
int getchar();
int putchar(int c);
int printf(const char *format , ... );
int scanf (const char *format , ... );
\end{lstlisting}
\end{frame}

\begin{frame}[fragile]\ft{函数类型}
  \begin{lstlisting}[language=c,backgroundcolor=\color{red!10}]
// math.h
double sin(double);   
double cos(double);   
double tan(double);   
double asin(double);  
double acos(double); 
double atan(double); 
double log(double);  
double log10(double); 
double pow(double x,double y); 
double exp(double); 
double sqrt(double); 
int abs(int);  
double fabs(double); 
\end{lstlisting}
\end{frame}
