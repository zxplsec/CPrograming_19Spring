\subsection{上机操作}

\begin{frame}[fragile]\ft{\subsecname:习题一}
\begin{enumerate}\setcounter{enumi}{0} 
\item 编写程序,输入名和姓,然后以“名, 姓”的格式打印。
\end{enumerate}

\begin{lstlisting}
Xiaoping Zhang
\end{lstlisting}
\end{frame}

\begin{frame}[fragile]\ft{\subsecname:习题一}

\end{frame}

\begin{frame}[fragile]\ft{\subsecname:习题二}
\begin{enumerate}\setcounter{enumi}{1} 
\item 编写程序,输入名字,并执行以下操作:\\[0.05in]
\begin{itemize}
\item 把名字括在双引号中打印出来\\[0.1in]
\item 在宽度为20个字符的字段内打印名字,并且整个字段括在引号内\\[0.1in]
\item 在宽度为20个字符的字段的左端打印名字,并且整个字段括在引号内\\[0.1in]
\item 在比名字宽3个字符的字段内打印它。
\end{itemize}
\end{enumerate}

\begin{lstlisting}[showspaces=true]
"Xiaoping"
"            Xiaoping"
"Xiaoping            "
"   Xiaoping"
\end{lstlisting}
\end{frame}

\begin{frame}[fragile]\ft{\subsecname:习题三}
\begin{enumerate}\setcounter{enumi}{2} 
\item 编写程序,读取一点浮点数,以如下方式打印:
\begin{lstlisting}[showspaces=true]
a. The input is 21.3 or 2.1e+001
b. The input is +21.290 or 2.129E+001
\end{lstlisting}
\end{enumerate}
\end{frame}


\begin{frame}[fragile]\ft{\subsecname:习题四}
\begin{enumerate}\setcounter{enumi}{3} 
\item 编写程序,要求输入身高(以cm为单位)和名字,然后以如下形式显示:
\begin{lstlisting}[showspaces=true]
Xiaoping, you are 1.70m tall.
\end{lstlisting}
使用float类型,使用/作为除号。
\end{enumerate}
\end{frame}


\begin{frame}[fragile]\ft{\subsecname:习题五}
\begin{enumerate}\setcounter{enumi}{4} 
\item 编写程序,首先输入名字,然后输入姓氏。在一行打印输入的姓名,在下一行打印每个名字中字母的个数,把字母个数与相应名字的结尾对齐。以如下形式显示:
\begin{lstlisting}[showspaces=true]
Xiaoping Zhang
       8     5
\end{lstlisting}
然后打印相同的信息,但是字母个数与相应单词的开始对齐。
\end{enumerate}
\end{frame}


\begin{frame}[fragile]\ft{\subsecname:习题六}
\begin{enumerate}\setcounter{enumi}{5} 
\item 编写程序,设置一个值为1.0/3.0的double类型变量和一个值为1.0/3.0的float类型变量。每个变量的值显示三次:\\[0.05in]
\begin{itemize}
\item 一次在小数点后显示4位;\\[0.1in]
\item 一次在小数点后显示12位;\\[0.1in]
\item 一次在小数点后显示16位。\\[0.05in]
\end{itemize}
同时让程序包含float.h文件,并显示FLT\_DIG和DBL\_DIG的值。

\end{enumerate}
\end{frame}
