
\section{常量与预处理器}

\begin{frame}[fragile]\ft{\secname}
\lstinputlisting
{slide04/code/circle1.c}
\end{frame}

% \begin{frame}[fragile]\ft{\secname}
% \begin{lstlisting}[backgroundcolor=\color{red!10}]
% radius =  1.000000, circum = 6.283185, area = 3.141593
% \end{lstlisting}
% \end{frame}

\begin{frame}[fragile]\ft{\secname}
\lstinputlisting
{slide04/code/circle2.c}
\end{frame}

\begin{frame}[fragile]\ft{\secname}
\lstinputlisting
{slide04/code/circle3.c}
\end{frame}

\begin{frame}[fragile]\ft{\secname}
\lstinputlisting
{slide04/code/circle4.c}

\end{frame}

\begin{frame}[fragile]\ft{\secname:宏定义}
  \begin{free}[宏定义的一般形式]{}
\begin{lstlisting}
#define NAME value
\end{lstlisting}
  \end{free} \pause 

  \begin{free}[注]{}
    \begin{itemize}
    \item 不是C语句,不需要分号;是一条预处理指令\\[0.1in]
    \item 宏名使用大写,以示跟普通变量的差别。\\[0.1in]
    \item 宏名的命名请遵循变量命名规则。
    \end{itemize}
  \end{free} \pause 

  \lstinline|#define| 也可用于定义字符和字符串变量,前者用 \lstinline|' '|,后者用 \lstinline|" "|。

\begin{lstlisting}
#define BEEP '\a'
#define TEE 'T'
#define ESC '\033'
#define OOPS "Now you have done it!"
\end{lstlisting}

\end{frame}

\begin{frame}[fragile]\ft{\secname:宏定义}

  \begin{free}[常见错误]{}
\begin{lstlisting}
#define B = 20
\end{lstlisting}
  \end{free}
  
\pause 

此时,\lstinline|B| 将会被 \lstinline|= 20| 而不是 \lstinline|20| 代替。 若使用以下语句
\begin{lstlisting}
c = a + B;
\end{lstlisting}
会被替换成如下错误的表达:
\begin{lstlisting}
c = a + = 20;
\end{lstlisting}
\end{frame}

\begin{frame}[fragile]\ft{\secname:函数宏}
  \begin{defn}[]{}
    在宏定义中,也可以像一个“函数”一样实现某种功能,这种用法叫\red{函数宏}。  
  \end{defn} \pause 

  \lstinputlisting[]{slide04/code/f_macro1.c}
  \pause 
  \begin{lstlisting}
max(1, 2) = 2
1.0 + max(1.1+2.2, 3.0) = 4.30
  \end{lstlisting}
\end{frame}

\begin{frame}[fragile]\ft{\secname:函数宏}

  错误的用法
  \lstinputlisting[]{slide04/code/f_macro2.c}
  \pause 
  \begin{lstlisting}
max(1, 2) = 2
1.0 + max(1.1+2.2, 3.0) = 3.30
  \end{lstlisting}
\end{frame}

\begin{frame}[fragile]\ft{\secname:函数宏}

  \lstinputlisting[]{slide04/code/f_macro3.c}
  \pause 
  \begin{lstlisting}
the square of 3 is 9.
the square of 3+2 is 25.
  \end{lstlisting}
\end{frame}

\begin{frame}[fragile]\ft{\secname:函数宏}
  错误的用法
  \lstinputlisting[]{slide04/code/f_macro4.c}
  \pause 
  \begin{lstlisting}
the square of 3 is 9.
the square of 3+2 is 11.
  \end{lstlisting}
\end{frame}


\begin{frame}[fragile]\ft{\secname:函数宏}

  \lstinputlisting[]{slide04/code/f_macro5.c}
  \pause 
  \begin{lstlisting}
3: 3
3: 3
3+2: 5
5: 5
  \end{lstlisting}
\end{frame}


\begin{frame}[fragile]\ft{\secname:函数宏}

  \lstinputlisting[]{slide04/code/f_macro6.c}
  \pause 
  \begin{lstlisting}
x1 = 12
x2 = 24
x3 = 36
\end{lstlisting}
\end{frame}

\begin{frame}\ft{\secname:函数宏}
  \begin{free}[注意]{}
    \begin{itemize}
    \item 宏名中不允许出现空格,因为宏定义把宏名后的第一个空格认作宏名与其替换值的分割符。
    \item 为了避免一些运算优先级的错误,请注意括号的使用。
    \item 可以跨行进行宏定义,请使用 \lstinline|\\| 断行。
    \end{itemize}
  \end{free}

  \pause 
  \begin{free}[\lstinline|\#|与\lstinline|\#|\lstinline|\#|在函数宏中的使用]{}  
    \begin{itemize}
    \item \lstinline|\#| 把宏参数转换为一个字符串。例如,若 \lstinline|x| 是一个宏参数,则 \lstinline|\#x| 可把参数转换为相应的字符串,该过程称为\red{字符串化}。
    \item \lstinline|\#|\lstinline|\#| 用于宏参数的连接。
      \begin{itemize}
      \item 若有\lstinline|\#define X(n)| \lstinline|x\#\#n|,则 \lstinline|X(2)| 将展开为 \lstinline|x2|.
      \item 若有\lstinline|\#define CONS(a,b)| \lstinline|(a\#\#e\#\#b)|,则 \lstinline|CONS(2,3)| 将展开为 \lstinline|2e3|.
      \end{itemize}
    \end{itemize}

  \end{free}
\end{frame}

\begin{frame}[fragile]\ft{\secname:const修饰符}
C90允许使用关键字const把一个变量声明转换为常量声明:
\begin{lstlisting}
const int MONTHS = 12; 
\end{lstlisting}
这使得MONTHS成为一个只读值。你可以显示它,并把它用于计算中,但不能改变它的值。
\end{frame}
