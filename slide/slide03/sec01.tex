\section{基本概念}

%\begin{frame}
%数据类型有两大系列:
%\begin{enumerate}
%\item 整数类型
%\item 浮点数类型
%\end{enumerate}
%
%本章将介绍这些数据类型以及如何声明它们、如何使用它们。
%\end{frame}

\begin{frame}\ft{常量}
\begin{defn}[\blue{常量}]{}
在程序执行过程中,值不发生改变的量称为常量。
\end{defn} \vspace{0.1in}

常量分为两类:\vspace{0.05in}

\begin{enumerate}
\item 直接常量(或字面常量)\\[0.1in]
\item 符号常量
\end{enumerate}
\end{frame}

\begin{frame}\ft{直接常量}

\begin{itemize}
\item 整型常量:12、0、-3;\\[0.1in]
\item 浮点型常量:3.1415、-1.23;\\[0.1in]
\item 字符型常量:'a'、'b'
\end{itemize}
\end{frame}

\begin{frame}[fragile]\ft{符号常量}

\begin{defn}[\blue{标识符}]{}
用来标识变量名、符号常量名、函数名、数组名、类型名、文件名的有效字符序列。
\end{defn} \pause \vspace{0.1in}

\begin{defn}[\blue{符号常量}]{}
在C/C++中,可以用一个标识符来表示一个常量,称之为符号常量。
\end{defn} 
\end{frame}

\begin{frame}[fragile]\ft{符号常量}

  符号常量在使用之前必须先定义,其一般形式为:
\begin{lstlisting}
#define `标识符` `常量`
\end{lstlisting}
\vspace{0.05in}

\begin{itemize}
\item  \lstinline|#define| 是一条预处理命令,称为\red{宏定义}。\\[0.1in]
\item 功能是把该标识符定义为其后的常量值。\\[0.1in]
\item 一经定义,在程序的预处理阶段该标识符会被替换成该常量值。
\end{itemize}
\end{frame}

\begin{frame}[fragile]\ft{符号常量}
\begin{lstlisting}
#include<stdio.h>
#define PRICE 100
int main(void)
{
  int num, total;  
  num = 10;
  total = num * PRICE;
  printf("total=%d\n", total);  
  return 0;
}
\end{lstlisting}

% \end{frame}

% \begin{frame}[fragile]\ft{符号常量}
\pause 
使用符号常量的好处是:\vspace{0.05in}

\begin{itemize}
\item 含义清楚;\\[0.1in]
\item 能做到“一改全改”。
\end{itemize}
\end{frame}


\begin{frame}\ft{变量}
\begin{defn}[\blue{变量}]{}
在程序执行过程中,值可以改变的量称为变量。
\end{defn}

\begin{itemize}
\item
一个变量应该有一个名字,在内存中占据一定的存储单元。\\[0.1in]
\item
变量定义必须放在变量使用之前。
\end{itemize}
\end{frame}


\begin{frame}\ft{变量}
\begin{itemize}
\item 在C中,变量的定义一般放在函数体的开头部分。\\[0.1in]
\item 在C++中,变量的定义可以放在任何位置。\\[0.1in]  
\item
要区分变量名和变量值是两个不同的概念。
\end{itemize}

\begin{figure}
\centering
\begin{tikzpicture}
\draw[thick] (0,0) rectangle (1,1);
\node at (0.5,0.5) [] {3};
\node at (0.5,1.3) [] {$a$};
\draw[->,>=stealth] (1.5,1.3) node[right]{\footnotesize{变量名}}--(0.7,1.3);
\draw[->,>=stealth] (1.5,0.5) node[right]{\footnotesize{变量值}}--(0.7,0.5);
\draw[->,>=stealth] (1.5,-.3) node[right]{\footnotesize{存储单元}}--(1.0,-.3)--(0.7,-0.05);
\end{tikzpicture}
\end{figure}

\end{frame}


% \begin{frame}\ft{数据类型}
% \begin{figure}
% \includegraphics[width=3.5in]{Fig/datatype}
% \end{figure}
% \end{frame}


\begin{frame}\ft{数据类型}
\begin{itemize}
\item 
对于常量,编译器通过书写形式来辨认其类型。\\[0.1in]
\item[] 例如,42是整型,42.0是浮点型。\\[0.2in]
\item
变量必须在声明语句中指定其类型。
\end{itemize}
\end{frame}

\begin{frame}\ft{数据类型关键字}
\begin{table}
\centering
\begin{tabular}{p{2cm}|p{2cm}|p{2cm}}\hline
\lstinline|int| & \lstinline|signed| & \lstinline|_Bool| \\[0.05in]
\lstinline|long| & \lstinline|void| & \lstinline|_Complex| \\[0.05in]
\lstinline|short| & & \lstinline|_Imaginary|\\[0.05in]
\lstinline|unsigned| &&\\[0.05in]
\lstinline|char| &&\\[0.05in]
\lstinline|float| &&\\[0.05in]
\lstinline|double| &&\\\hline
\end{tabular}
\end{table}
\end{frame}

\begin{frame}\ft{数据类型关键字}
\begin{itemize}
\item 整型
  \begin{itemize}
  \item \lst|int|
  \item \lst|long|、\lst|short|、\lst|unsigned| 和 \lst|signed|
  \end{itemize}
\item 字符型:
  \begin{itemize}
  \item \lst|char|
  \end{itemize}
\item 浮点型
  \begin{itemize}
  \item \lst|float|、\lst|double| 和 \lst|long double| 
  \end{itemize}
\item 布尔型:(布尔值为\lst|true| 和 \lst|false|)
  \begin{itemize}
  \item C:\lst|_Bool|
  \item C++: \lst|bool|
  \end{itemize}
\item 复数: 
  \begin{itemize}
  \item C: \lst|complex|
  \item C++:\lst|complex| 
  \end{itemize}
\end{itemize}
\end{frame}

\begin{frame}{布尔类型}
  \lstinputlisting[basicstyle=\ttfamily\small]{slide03/code/bool.c} \pause 
  \lstinputlisting[basicstyle=\ttfamily\small]{slide03/code/bool.cpp}
\end{frame}

\begin{frame}{复数类型}
  \begin{table}
    \centering
    \caption{C风格的复数类型及其操作,需要用到头文件 \lst|complex.h|}
    \begin{tabular}{p{2cm}|p{6cm}}\hline
      函数&功能\\\hline
      \lst|creal| & 获取复数的实部\\
      \lst|cimag| & 获取复数的虚部\\
      \lst|conj|  & 获取复数的共轭\\
      \lst|cabs|  & 获取复数的模\\
      \lst|carg|  & 获取幅角\\\hline
    \end{tabular}
  \end{table}
\end{frame}

\begin{frame}{复数类型}
  \lstinputlisting[basicstyle=\ttfamily\small]{slide03/code/complex.c}
\end{frame}


\begin{frame}{复数类型} 
  \begin{table}
    \centering
    \caption{C++风格的复数类型及其操作,需要用到头文件 \lst|complex|}
    \begin{tabular}{p{2cm}|p{6cm}}\hline
      方法&功能\\\hline
      \lst|real| & 获取复数的实部\\
      \lst|imag| & 获取复数的虚部\\
      \lst|conj| & 获取复数的共轭\\
      \lst|abs|  & 获取复数的模\\
      \lst|arg|  & 获取复数的幅角\\\hline
    \end{tabular}
  \end{table}
\end{frame}


\begin{frame}{复数类型} 
  \lstinputlisting[basicstyle=\ttfamily\small]{slide03/code/complex.cpp}
\end{frame}



